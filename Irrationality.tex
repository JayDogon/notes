\section{Irrationality}

\subsection{Algebraic Numbers}
\begin{theorem}\label{RRT}
    If $x = a/b$, where $a$ and $b$ are coprime, is a rational root of the equation

    \begin{equation*}
        c_0 x^m + c_1x^{m-1} + ... + c_m = 0
    \end{equation*}
    \noi
    where the $c_i$ are integral, then $b|c_0$ and $a|c_m$.
\end{theorem}

\begin{proof}
    WLOG we may assume $c_0$ and $c_m$ are not 0. By substituting in the given root and clearing 
    denominators we obtain the following.

    \begin{equation*}
        c_0 a^m = b(-c_1 a^{m-1} - c_2 a^{m-2}b - ... - c_m b^{m-1})
    \end{equation*}
    Therefore, $b|c_0 a^m$. Because $b$ is coprime with $a^m$, it follows that $b$ divides $c_0$. \\
    In much the same way we see that 
    \begin{equation*}
        c_m b^m = a(- c_{m-1}b^{m-1} - c_{m-2}b^{m-2}a - ... - c_0 a^{m-1})
    \end{equation*}
    from which the remaining claim follows.
\end{proof}
This provides an effective algorithm for determining the rationality of algebraic numbers. In particular, 
one can look at all combinations of numerators and denominators which divide the leading and constant 
coefficients respectively.
\subsection{Transcendental Numbers}
If a number is not prescribed in an algebraic manner in this way it can often be difficult to discern 
whether or not it's even rational. In this section we prove some classic irrationality results for 
some numbers that are in fact known to be transcendental.



\subsection{Exercises}
\begin{enumerate}
    \item Show that $\sqrt[m]{N}$ is irrational or integral. 
\end{enumerate} 




