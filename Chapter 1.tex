\section{Notation and introductory concepts}

\subsection{$a|b$}

We will denote by $a|b$, where $a$ and $b$ are understood to be positive integers, that $a$ divides $b$. That is, 
there exists some integer $m$ such that $am = b$. \\ \\

\subsection{Big O}

\noi
Suppose $\phi$ is some real valued function on a particular domain. Then by $O(\phi)$ we denote the class of complex valued 
functions $f$ such that there exists a constant $A$ with

\begin{equation*}
    |f| < A\phi
\end{equation*}

\noi
over the entirety of the domain.

\subsection{Little o, $\prec$}
Suppose $\phi$ is as before. Then by $o(\phi)$ we denote the class of functions $f$ with

\begin{equation*}
    f/\phi \to 0
\end{equation*}
\noi
By $f \prec \phi$ we mean that $f \in o(\phi)$

\subsection{$\sim$}
By $f \sim \phi$ we mean that

\begin{equation*}
    f/\phi \to 1
\end{equation*}