\section{Solutions}

\subsection{Primes}

\subsubsection{2.1}
The existence of prime decompositions is a simple matter 
of induction. Let $p_1 p_2 ... p_m$ and $q_1q_2...q_n$ be 
two equal prime decompositions. Then $p_1$ divides the 
product of the $q_i$. By Euclid's Lemma, it must divide 
some $q_j$. But then it follows that $p_1 = q_j$. Hence

\begin{equation*}
    p_2 p_3 ... p_m = q_1 ... q_{j-1}q_{j+1} ... q_m
\end{equation*}

One may continue this process until either one or both of 
the products is empty. If only one is empty, then it must 
be that a product of primes is equalt to 1, which is 
clearly impossible. If they are both empty, then then 
the products were identical.

\subsubsection{2.2}

Suppose there are only finitely many primes of the form 
$6n+5$. Let $q = 2 \cdot 3 \cdot ... \cdot p - 1$ where 
$p$ is the largest prime of the form $6n+5$. $q$ is 
clearly of the form $6n+5$, but then it must be divisible 
by at least one prime of the form $6n+5$ since a product 
of the other remainders cannot be of this form. 
$q$ is clearly coprime to all of the primes of this form. 
We reach the familiar contradiction.

\subsubsection{2.3}
Suppose $n>m\geq 1$

\begin{align*}
    F_n - F_m &= 2^{2^n} - 2^{2^m} \\
    &= 2^{2^m}(2^{2^m(2^{n-m} - 1)}) - 1 
\end{align*}

If $p$ is a common prime divisor of $F_n$ and $F_m$ 
it must be odd and it must divide their difference. 
Therefore, it must divide $2^{2^m(2^{n-m} - 1)} - 1$. 
Let $p$ be an odd prime divisor of $2^{2^m} + 1$. Then 
$2^{2^m}$ is congruent to $-1$ mod $p$. Because $n-m$ 
is at least 1, $2^{n-m} - 1$ is odd. But then 
$2^{2^m(2^{n-m} - 1)} = \left(2^{2^m}\right)^{2^{n-m}-1}$ 
is congruent to $-1$ mod $p$. But then it follows that 
$F_n - F_m$ is not divisible by $p$. \\ \\

If the distinct $F_n$ are all coprime, then each possesses 
a unique set of prime divisors and the infinitude of primes 
follows.

\subsubsection{2.4}

If $a$ were odd then $a^n + 1$ would be even and greater than 
2. Suppose $n$ had an odd divisor $d$ with $n = dq$. Then,

\begin{align*}
    a^{dq} + 1 &= \left(a^{q}\right)^d + 1 \\
    &= (a^q + 1)(a^{q(d-1)} - a^{q(d-2)} + ... + 1)
\end{align*}

but this is a non trivial factorisation of $a^n + 1$.

\subsubsection{2.5}

\begin{equation*}
    a^n - 1 = (a-1)(a^{n-1} + a^{n-2} ... + 1)
\end{equation*}

If $a > 2$ this factorisation is immediately non trivial. 
Suppose $a=2$ and $n = qd$ is not prime.

\begin{align*}
    2^{qd} &= \left(2^q\right)^d - 1 \\
    &= (2^{q} - 1)(2^{q(d-1)} + 2^{q(d-2)} + ... + 1)
\end{align*}

\subsubsection{2.6}

Suppose $f(n) = \sum_{k=0}^m c_k n^k $ is a polynomial with 
integral coefficients. Then

\begin{align*}
    f(tc_0) &= c_0 + \sum_{k=1}^m c_k t^k c_0^k \\
    &= c_0(1 + \sum_{k=1}^m c_k t^k c_0^{k-1})
\end{align*}

If $c_0\neq \pm 1$ this provides an immediate non trivial 
factorisation for large $t$. Suppose $c_0 = 1$. Then 

\begin{equation*}
    f(n) = \pm 1 + \sum_{k=1}^m c_k n^k
\end{equation*}